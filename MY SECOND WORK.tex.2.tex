\documentclass[18pt,a4paper]{article}
\usepackage{}
\begin{document}

\title{GOMBE STATE UNIVERSITY, BOMBE, NIGERIA. 
\\DEPARTMENT OF MATHEMATICS
\\. PG23/MSC/MATH/1014
\\. SECOND ASSIGMENT MATH 816}
\maketitle
\newpage
\section{GROUP THEORY: Introduction }
In modern algebra, the study of groups, which are systems consisting of a set of elements and a binary operation that can be applied to two elements of the set, which together satisfy certain axioms. These require that the group be closed under the operation (the combination of any two elements produces another element of the group), that it obey the associative law, that it contain an identity element (which, combined with any other element, leaves the latter unchanged), and that each element have an inverse (which combines with an element to produce the identity element). If the group also satisfies the commutative law, it is called a commutative, or abelian, group. The set of integers under addition, where the identity element is 0 and the inverse is the negative of a positive number or vice versa, is an abelian group.


\section{Definition: group theory}
Group theory in mathematics refers to the study of a set of different elements present in a group. A group is said to be a collection of several elements or objects which are consolidated together for performing some operation on them. Inset theory, you have been familiar with the topic of sets. If any two of the elements of a set are combined through an operation for producing a third element that belongs to the same set and that meets the four hypotheses that are the closure, the associativity, the invertibility, and the identity, they are referred to as group axioms. A group of integers is performed under the multiplication operation. Geometric group theory according to the branch of mathematics refers to the study of the groups which are finitely produced by using the research of the relationships between the different algebraic properties of these groups and the topological and the geometric properties of space. In this article, we will learn about what group theory is, what are the applications of group theory in mathematics and look at some group theory examples.


 
\subsection{Algebraic groups}
The equations defining a unitary group are polynomial equations over k (but not over K): for the standard form 

\section{Properties of Group Theory}
Let us learn about group theory math properties.
Consider dot (.) to be an operation and G to be a group. The axioms of the group theory are defined in the following manner:
Closure: If x and y are two different elements in group G then x. y will also be a part of group G
Associativity: If x, y, and z are the elements that are present in group G, then you get x. (y. z) = (x . y) . z.
Invertibility: For every element x in the group G, there exists some y in the group G in a way that; x. y = y . x.
Identity:  For any given element x in group G, there exists another element called I in group G in a way that x. I = I . x, wherein I refers to the identity. element of group G

\section{Applications of Group Theory}
Let us now look at what are the applications of group theory in mathematics
Let us learn about group theory math properties.
Consider dot (.) to be an operation and G to be a group. The axioms of the group theory are defined in the following manner:
Closure: If x and y are two different elements in group G then x.y will also be a part of group G.
Associativity: If x, y, and z are the elements that are present in group G, then you get x. (y. z) = (x . y) . z.
Invertibility: For every element x in the group G, there exists some y in the group G in a way that; x. y = y . x.
Identity:  For any given element x in group G, there exists another element called I in group G in a way that x. I = I . x, wherein I refers to the identity element of group G.

\section{Applications of Group Theory}

Let us now look at what are the applications of group theory in mathematics. In Mathematics and abstract algebra, group theory studies the algebraic structures that are called groups. The concept of the group is a center to abstract algebra. The other well-known algebraic structures like the rings, fields and vector spaces are all seen as the groups that are endowed with the additional operations and axioms. Groups recur throughout when it comes to mathematics, and the methods of group theory have influenced several parts of algebra. The linear algebraic groups and the Lie groups are the two branches of group theory that have experienced advances and are the subject areas in their own ways.
Hence the group theory and the closely related theory called the representation theory to have several important applications in the fields of physics, material science, and chemistry. The group theory is also the center of public-key cryptography.
\subsection{Group Theory Example}
 Example:1 Let G be a group.
 Prove that the element  e$ \in$Gis unique . Also  prove that each of the elements e$\in$G  consists of a unique inverse which is denoted by 
\\Solution: 
\\Consider e and e’ to be the identities. 
According to the definition,
 you get
\\ e' = e * e' = e.
\\Similarly, consider y and y' to be the inverses of x. 
\\Then, you would get
\\y = y * e
$\\ = y * (x * y’)$
$\\ = (y * x) * y’$
$\\= e * y’$
$\\= y’$


Example 2: Consider x,  y$\in$G having the x-1  and y-1   respectively. Determine the inverse of xy. 
\\Solution:
\\  The inverse of the product of x and y is given as follows:
$\\ x * yy = x-1 * y-1$
\\You have
$\\ (x * y) * (x-1 * y-1)$
$\\ = x (y * y-1) x-1$
$\\ = xex-1$
$\\ = e$
\\Similarly,
$\\ (x-1 * y-1) * (x * y)$
$\\= e$
\\Therefore,
$\\ (xy)-1$
$\\ = x-1$
\section{Pr
operties of Group}
Theory If Dot(.) is an operation and G is a group, then the axioms of group theory is defined as;
\\•	Closure: If x, y are elements in a group, G, then x.y is also an element of G.
\\•	Associativity: If x, y and z are in group G, then x . (y . z) = (x . y) . z.
\\•	Invertibility: For every x in G, there is some y in G, such that; x. y = y . x.
\\•	Identity: For any element x in G, there is an element I in G, such that: x. I = I . x, where I is the identity of G.

\section{Applications of Group Theory}
The following are some of the important applications of Group Theory
\\•	 If an object or a system property is invariant under the transformation, the object can be analyzed using group theory, because group theory is the study of symmetry.
\\•	Rubik’s cube can be solved using the algorithm of group theory.
\\•	Modeling of the crystals and the hydrogen atom are done using symmetry groups
\\  •	Many fundamental laws of nature in Physics, Chemistry, and Material science use symmetry. 
\section{Representations of finite groups }
Category of group representations. Let G be a group. Let V be a vector space over C. Denote by GL(V ) the general linear group of V , i.e., the group of all linear automorphisms of V . A representation ($\pi$, V ) of G on the vector space V is a group homomorphism $\pi$ : G $\rightarrow$GL(V ). A morphism $\phi$ : ($\pi$, V) $\rightarrow$ (v,U) of representation ($\pi$,V ) into (µ,U) is a linear map $\phi$ : V $\rightarrow$ U such that the diagram 
 \section{diagram}

commutes for all g$\in$ G. Morphisms of representations are also called intertwining maps. The set of all morphisms of      ($\pi$ , V ) into (v U) is denoted by Hom G(V, U).
    It is easy to check that all representations of G form a category Rep(G) of representations of G. 
An isomorphism $\phi$ : ($\pi$, V ) $\rightarrow$ (v, U) in this category is a morphism of representations which is a linear isomorphism of the vector space V with U. If two representations are isomorphic, we say that they are equivalent.
     Let ($\pi$, V ) and (v, U) be two representations of G. Let $\phi$ and $\psi$ be two morphisms in HomG(V, U Then $\phi$ + $\psi$ is also a morphism of ($\pi$, V ) into (v, U), and HomG(V, U) is an abelian group.

\subsection{ Representations of finite groups.}
 Let G be a group. We say that G is a finite group, if G is a finite set. we assume that the group G is finite. We put [G] = Card(G). 
    A representation ($\pi$, V ) of G is finite-dimensional if V is a finite-dimensional vector space. We put dim $\pi$ = dimC V.
\\ Lemma.  
\\Let ($\pi$, V ) be a representation of G. Let v$\in$ V , v 6= 0. Then there exists a finite-dimensional sub representation (v, U) of ($\pi$, V ) such that v$\pi$ U. 
\\   Proof.
\\ Let U be the vector subspace of V generated by vectors $\pi$(g)v, g $\in$ G. Then U is G-invariant and finite-dimensional.
 Moreover, v == $\pi$(1)v is in U.
 A representation ($\pi$, V ) of G is called irreducible if the only G-invariant subspaces in V are {0}  
\subsection{Regular representation.}
 Let G be a finite group. Denote by C[G] the space of all complex valued functions on G. clearly, dim C[G] = [G]. 
The vector space C[G] has a structure of inner product space with the inner product (f | f ' ) = 1 [G] X g$\in$G 
\\ for f,f' $\in$ [G]. for g $\in$ G and f $\in$ C[G] define the function R(g)f by (R(g)f)(h) = f(gh) forany h $\in$ G. Clearly, R(g) : f 7$\rightarrow$ R(g)f is a linear map on C[G]. Moreover, for g,h $\in$ G, we have 
\\(R(gh)f)(k) = f(kgh) = (R(h)f)(kg) = (R(g)R(h)f)(k) for any k $\in$ K.
\\ Therefore R(gh) = R(g)R(h). Clearly, R(1) = I
 It follows that (R,C[G]) is a representation of G. We call it the (right) regular representation of G.


\end{document}